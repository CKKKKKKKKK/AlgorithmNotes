\documentclass[11pt]{article}
\usepackage{latexsym}
\usepackage{amsmath,amssymb,amsthm}
\usepackage{epsfig}
\usepackage[right=0.8in, top=1in, bottom=1.2in, left=0.8in]{geometry}
\usepackage{setspace}

\spacing{1.06}

\newcommand{\handout}[5]{
  \noindent
  \begin{center}
  \framebox{
    \vbox{\vspace{0.25cm}
      \hbox to 5.78in { {EI6303:\hspace{0.12cm}Design and Analysis of Algorithms} \hfill #2 }
      \vspace{0.48cm}
      \hbox to 5.78in { {\Large \hfill #5  \hfill} }
      \vspace{0.42cm}
      \hbox to 5.78in { {#3 \hfill #4} }\vspace{0.25cm}
    }
  }
  \end{center}
  \vspace*{4mm}
}
\newcommand{\lecture}[4]{\handout{#1}{#2}{#3}{Scribes:\hspace{0.08cm}#4}{Notes #1}}
\newtheorem{theorem}{Theorem}
\newtheorem{corollary}[theorem]{Corollary}
\newtheorem{lemma}[theorem]{Lemma}
\newtheorem{observation}[theorem]{Observation}
\newtheorem{example}[theorem]{Example}
\newtheorem{definition}[theorem]{Definition}
\newtheorem{property}[theorem]{Property}
\newtheorem{claim}[theorem]{Claim}
\newtheorem{fact}[theorem]{Fact}
\newtheorem{assumption}[theorem]{Assumption}
\newcommand{\E}{\textbf{E}}
\newcommand{\var}{\text{var}}
\def\eps{\ensuremath\epsilon}
\begin{document}

\lecture{1 -- Median trick}{May 14, 2020}{Instructor:\hspace{0.08cm}\emph{Guoqiang Li}}{\emph{Yifan Zhou}}


\section{Simple Unit-Capacity Networks}

\begin{definition}
A flow network is a simple unit-capacity network if:
\begin{itemize}
  \item Every edge has capacity 1.
  \item Every node (other than s or t) has exactly one entering edge, or exactly one leaving edge, or both (as depicted in figure~\ref{fig:simple_unit_capacity}).
\end{itemize}
\end{definition}

\begin{figure}
  \centering
  \includegraphics[width=0.4\textwidth]{images/simple_unit_capacity.png}
  \caption{Three cases of nodes in a simple unit-capacity network}
  \label{fig:simple_unit_capacity}
\end{figure}

Also we have the following property:

\begin{property}
Let $G$ be a simple unit-capacity network and let $f$ be a 0-1 flow. Then, residual network $G_f$ is also a simple unit-capacity network.
\end{property}

This property is very obvious.

\subsection{Maximum Flow in Simple Unit-Capacity Networks}

To compute a maximum flow in simple unit-capacity networks, we can use Dinitz' algorithm.

\begin{theorem}[Even-Tarjan 1975] \label{dinitz in simple unit-capacity networks}
In simple unit-capacity networks, Dinitz’algorithm computes a maximum flow in $O(|E||V|^{1/2})$ time.
\end{theorem}

Here we review the Dinitz' algorithm first. Phase of normal augmentations is:
\begin{itemize}
  \item Construct level graph $L_G$ according to the graph $G$.
  \item Start at $s$ and advance along an edge in $L_G$ until reach at $t$ or get stuck.
  \item If reach $t$, augment the flow, update $L_G$ and restart from $s$.
  \item If get stuck, then delete current node from $L_G$ and retreat to previous node.
\end{itemize}

To prove this theorem, we have to prove three lemmas first:
\begin{itemize}
  \item Lemma 1. Each phase of normal augmentations takes $O(|E|)$ time.
  \item Lemma 2. After $|V|^{1/2}$ phases, $val(f)\ge val(f^*)-|V|^{1/2}$. $val(f)$ is the value of current flow, and $val(f^*)$ is the value of the optimal flow.
  \item Lemma 3. After at most $|V|^{1/2}$ additional augmentations, the flow is optimal.
\end{itemize}

\begin{proof}
First we can prove lemma 1:
\begin{itemize}
  \item We run BFS algorithm to create the level graph. BFS takes $O(|E| + |V|)$ time. Because we suppose that $|E| >> |V|$, it takes $O(|E|)$ time to create level graph.
  \item In a simple unit-capacity network, each edge will be involved in at most one advance, retreat and augmentation operation per phase. So each edge takes $O(1)$ time, totally $O(|E|)$ time.
  \item In a simple unit-capacity network, each node will be involved in at most one retreat operation per phase. So each node takes $O(1)$ time, totally $O(|V|)$ time.
  \item In conclusion, each phase of normal augmentations takes $O(|E|)$ time.
\end{itemize}
Then we prove lemma 2:
\begin{itemize}
  \item After $|V|^{1/2}$ phases, the length of shortest augmentation path is greater than $|V|^{1/2}$, so the level graph has at least $|V|^{1/2}$levels (not including $s$ and $t$).
  \item Then there exists a level $h$, with number of nodes $|V_h| \le |V|^{1/2}$, as depicted in figure~\ref{fig:level_h_in_level_graph}. Otherwise the total number of nodes is greater than $|V|$.
  \item Let set $A=\{v:l(v)<h\} \cup \{v:l(v)=h\ and\ v\ has\ \le 1\ outgoing\ residual\ edge\}$.
  \item Then $cap_f(A,B)\le |V_h|\le |V|^{1/2}$, and we can get that $val(f^*) \le val(f) + cap_f(A,B) \Rightarrow val(f^*) \le val(f) + |V|^{1/2} \Rightarrow val(f) \ge val(f^*)-|V|^{1/2}$.
\end{itemize}
Finally, we prove lemma 3:
\begin{itemize}
  \item According to lemma 2, after $|V|^{1/2}$ phases, value of current flow plus $|V|^{1/2}$ is no less than value of the optimal flow.
  \item In a simple unit-capacity network, each augmentation increases flow value by at least 1.
  \item So after at most $|V|^{1/2}$ additional augmentations, the flow is optimal.
\end{itemize}
By proving the three lemmas, we can prove the theorem \ref{dinitz in simple unit-capacity networks} that in simple unit-capacity networks, Dinitz' algorithm computes a maximum flow in $O(|E||V|^{1/2})$ time. 
\end{proof}

\begin{figure}
  \centering
  \includegraphics[width=0.4\textwidth]{images/level_h_in_level_graph.png}
  \caption{Level h in the level graph}
  \label{fig:level_h_in_level_graph}
\end{figure}

\subsection{Compute Maximum-Cardinality Bipartite Matching by Dinitz' Algorithm}

Also we can use Dinitz' algorithm to compute maximum-cardinality bipartite matching:

\begin{corollary}
Dinitz' algorithm computes maximum-cardinality bipartite matching in $O(|E||V|^{1/2})$ time.
\end{corollary}

\begin{proof}
\ 
\begin{itemize}
  \item According to corollary before, we can solve bipartite matching problem via max-flow formulation.
  \item If we assign unit capacity to edges between $L$ and $R$, then the generated digraph is a simple unit-capacity network as depicted in figure~\ref{fig:unit_capacity_bipartite}.
  \item According to theorem \ref{dinitz in simple unit-capacity networks}, we can use Dinitz' algorithm to compute maximum-cardinality bipartite matching in $O(|E||V|^{1/2})$ time.
\end{itemize}
\end{proof}

\begin{figure}
  \centering
  \includegraphics[width=0.4\textwidth]{images/unit_capacity_bipartite.png}
  \caption{Generated simple unit-capacity network of bipartite}
  \label{fig:unit_capacity_bipartite}
\end{figure}

\section{Edge-Disjoint Paths}

\begin{definition}[Edge-Disjoint Paths]
Two paths are edge-disjoint if they have no edge in common.
\end{definition}

\subsection{Edge-Disjoint Paths Problem}

A \textbf{edge-disjoint paths problem} is to find the max number of edge-disjoint $s \leadsto t$ paths given a digraph $G=(V, E)$ and two nodes $s$ and $t$.

To solve the edge-disjoint paths problem, we can use max-flow formulation by assigning unit capacity to every edge of $G$, and calling the new graph $G'$.

\begin{theorem} \label{edge-disjoint paths and max flow}
There is 1-1 correspondence between $k$ edge-disjoint $s \leadsto t$ paths in $G$ and integral flows of value $k$ in $G'$.
\end{theorem}

\begin{proof}
$\rightarrow$:
\begin{itemize}
  \item Let $P_1,...,P_k$ be $k$ edge-disjoint $s \leadsto t$ paths in G.
  \item We can set $f(e)= \begin{cases} 1,\ edge\ e\ participates\ in\ some\ path\ P_j\\  0,\ otherwise\\ \end{cases}$
  \item Since paths are edge-disjoint, $f$ is a flow of value $k$.
\end{itemize}
$\leftarrow$:
\begin{itemize}
  \item Let $f$ be an integral flow in $G'$ of value $k$.
  \item We can consider edge $(s,u)$ with $f(s,u)=1$.
  \begin{itemize}
    \item By flow conservation, there exists an edge $(u,v)$ with $f(u,v)=1$.
    \item We can continue this process until reach $t$ by always choosing a new edge.
  \end{itemize}
  \item By repeating $k$ times, we can produce $k$ edge-disjoint $s \leadsto t$ paths.
\end{itemize}
In conclusion, there is 1-1 correspondence between $k$ edge-disjoint $s \leadsto t$ paths in G and integral flows of value k in G'.
\end{proof}

After prove theorem~\ref{edge-disjoint paths and max flow}, we can get a corollary:

\begin{corollary} \label{edge-disjoint paths solution corollary}
We can solve edge-disjoint paths problem via max-flow formulation.
\end{corollary}

\begin{proof}
\ 
\begin{itemize}
  \item According to the integrality theorem, there exists a max flow $f*$ in $G'$ that is integral.
  \item From the 1-1 correspondence proved above, we can know that $f*$ corresponds to max number of edge-disjoint $s \leadsto t$ paths in $G$.
  \item So we can solve edge-disjoint paths problem via max-flow formulation.
\end{itemize}
\end{proof}

\subsection{Network Connectivity}

\begin{definition}[Disconnect]
A set of edges $F \subseteq E$ disconnects $t$ from $s$ if every $s \leadsto t$ path uses at least one edge in $F$.
\end{definition}

A \textbf{network connectivity problem} is to find the minimal number of edges whose removal disconnects $t$ from $s$ given a digraph $G=(V,E)$ and two nodes $s$ and $t$. According to Menger's theorem, we can reduce the network connectivity problem to a edge-disjoint paths problem:

\begin{theorem}[Menger 1927]
The max number of edge-disjoint $s \leadsto t$ paths equals the mininal number of edges whose removal disconnects $t$ from $s$.
\end{theorem}

\begin{proof}
First we prove that the max number of edge-disjoint $s \leadsto t$ paths is no greater than the min number of edges whose removal disconnects $t$ from $s$:
\begin{itemize}
  \item We can suppose that the removal of $F \subseteq E$ disconnects $t$ from $s$, and $|F| = k$.
  \item Every $s \leadsto t$ path uses at least one edge in $F$, otherwise the removal of $F$ does not disconnect $t$ from $s$.
  \item Hence, the number of edge-disjoint path is no greater than $k$.
\end{itemize}
Second we prove that the max number of edge-disjoint $s \leadsto t$ paths is no less than the min number of edges whose removal disconnects $t$ from $s$:
\begin{itemize}
  \item Suppose max number of edge-disjoint $s \leadsto t$ paths is $k$ and let $F$ be the set of edges from set of nodes $A$ to set of nodes $B$.
  \item According to the theorem \ref{edge-disjoint paths and max flow} proved before, the value of max flow is $k$.
  \item According to the max-flow min-cut theorem, there exists a cut $(A, B)$ of capacity $k$.
  \item Then $|F| = k$ and disconnects $t$ from $s$.
\end{itemize}
In conclusion, the max number of edge-disjoint $s \leadsto t$ paths equals the mininal number of edges whose removal disconnects $t$ from $s$.
\end{proof}


\end{document}
